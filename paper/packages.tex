%% paragraphs within margins
\usepackage[english]{babel}
\setlength{\emergencystretch}{2pt}

% Math notation
\RequirePackage[group-separator={,}]{siunitx}
\newcommand{\na}{$\mathrm{N.A.}$}
\newcommand{\minus}{\scalebox{0.7}[1.0]{$-$}}
\newcommand{\expn}[1]{\!\!\times\!\!10^{\minus#1}}
\newcommand{\pval}[2]{$p\textrm{-value} = #1\expn{#2}$}
\newcommand{\set}[1]{\{{}#1\}}
\newcommand{\A}[1]{\mathcal{A}_{#1}}
\newcommand{\N}[1]{n_{#1}(j)}

%%%%%%%%%%% packages
\usepackage{caption}
\captionsetup{font={small}}
%\usepackage{cite}
%\usepackage{subfigure}
\usepackage{float}
\usepackage{listings}
\usepackage{caption}
\usepackage{graphicx}
\usepackage{subcaption}
\usepackage{colortbl}
\usepackage{amsmath} 
\usepackage{mathpartir}
\usepackage{fancyvrb}\fvset{fontsize=\small}
\usepackage{xspace}
\usepackage{xcolor}
\usepackage{balance}
\usepackage{wrapfig}
\usepackage{multirow}
\usepackage{pifont}
\usepackage{amssymb}
%\usepackage{amsfonts}
%%%%%%%%%%%%%%%%%%% this can reduce space quite a lot
\usepackage{soul}
\usepackage{microtype}
\usepackage[T1]{fontenc}
\usepackage{dsfont}
%%%%%%%%%%%%%%%%%%% this can reduce space quite a lot
\usepackage{relsize}
\usepackage{tikz}
\usepackage{pgfplots}
\usepackage{pgf-pie}
\usepackage{longtable}
\usepackage{booktabs}
\usepackage{marvosym} 
\usepackage{framed}
\usepackage{mdwlist}
\usepackage{tabularx}
\usepackage{wrapfig}
%% \usetikzlibrary{matrix,fit,shapes,calc,positioning,shadows,arrows,shapes,backgrounds,decorations.markings,fadings}
%% \tikzset{
%%     %Define standard arrow tip
%%     >=stealth',
%%     %Define style for boxes
%%     punkt/.style={
%%            rectangle,
%%            rounded corners,
%%            draw=black, very thick,
%%            text width=6.5em,
%%            minimum height=2em,
%%            text centered},
%%     % Define arrow style
%%     pil/.style={
%%            ->,
%%            thick,
%%            shorten <=2pt,
%%            shorten >=2pt,},
%%     % call out
%%     notice/.style  = { draw, rectangle callout, callout relative pointer={#1} }
%% }
\usepackage{array}
\usepackage{mathtools}
\usepackage{url}
\usepackage{color}

%% Colors
\definecolor{bgBlock}{rgb}{0.22,0.15,0.49}
\definecolor{bgBlockAlert}{rgb}{0.99,0.84,0.31}
\definecolor{fgBlockAlert}{rgb}{0.22,0.15,0.49}
\definecolor{fgBlock}{rgb}{0.99,0.84,0.31}
\definecolor{darkred}{rgb}{0.5,0,0}
\definecolor{darkgreen}{rgb}{0,0.5,0}
\definecolor{darkblue}{rgb}{0,0,0.5}
\definecolor{Gray}{gray}{0.9}
%\usepackage[bookmarks=false]{hyperref}
%% \hypersetup{colorlinks,
%%   linkcolor=darkblue,
%%   filecolor=darkgreen,
%%   urlcolor=darkred,
%%   citecolor=darkblue }



%%%%%%%%%%%%% code listing
\renewcommand{\ttdefault}{pcr}
\lstset{
  basicstyle=\scriptsize\ttfamily,
  keywordstyle=\scriptsize\ttfamily\bfseries,
  language=Java,             % choose the language of the code
  frame=single,              % adds a frame around the code
  aboveskip=0pt,
  belowskip=0pt,
  breaklines=true,           % sets automatic line breaking
  breakatwhitespace=false,   % sets if automatic breaks should only happen at
  showspaces=false,
  %numbersep=5pt,              % Abstand der Nummern zum Text
  %tabsize=2,                  % Groesse von Tabs
  %extendedchars=true,         %
  %breaklines=true,            % Zeilen werden Umgebrochen
  keywords=[2]{class, incorporateUserFeedback, testPushPop, testPopPush},
}
